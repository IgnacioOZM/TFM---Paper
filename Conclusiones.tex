\chapter{Conclusiones y futuros desarrollos}
\label{chap:Conclusiones y futuros desarrollos}
El desarrollo de este proyecto se caracteriza por cumplir dos grandes objetivos/metas. En primer lugar, su desarrollo ha permitido la creación de un sistema basado en redes neuronales capaz de identificar piezas y sus puntos de agarre óptimos. Y en segundo lugar ha permitido comprobar las capacidades de aprendizaje de una red neuronal al emplear un \textit{dataset} sintético. Es por ello que se debe de analizar el sistema desde las dos perspectivas.

\section{Dataset sintético}
\label{chap:Conclusiones sec:Dataset sintético}
Para el desarrollo del sistema de identificación de puntos de agarre se ha tenido que desarrollar un \textit{dataset} sintético que refleje la realidad y permita aumentar el número de imágenes para el entrenamiento. Este \textit{dataset} se ha realizado con blender y con la ayuda de blenderproc de forma que se ha conseguido generar miles de imágenes fotorealistas que representan y reflejan la zona de trabajo sobre la que se implantará el sistema.

Durante el desarrollo de este sistema se presto especial atención en representar lo mejor posible la realidad y el entorno de trabajo. De esta forma varios de los escenarios y de las configuraciones usadas se limitan a representar los escenarios observados en la planta de ensamblaje. Esta decisión desgraciadamente ha conllevado la generación de un \textit{dataset} sesgado en donde la mayoría de las piezas se encuentran en una posición predefinida por el escenario. Esto ha conllevado a que la red neuronal tienda a asumir que las piezas se encuentran en una de esas posiciones predefinidas y empeore los resultados del sistema.

Es por ello por lo que se recomienda mejorar la aleatoriedad del sistema introduciendo mas escenarios que a pesar de que no reflejen la realidad si permitan la obtención de imágenes con mayor riqueza y eviten el sesgado del la red. Este tipo de escenarios se deben de caracterizar por no presentar un suelo plano sino un suelo con irregularidades que fuercen a las piezas a nuevas posiciones.

\section{Sistema de detección de puntos de agarre}
\label{chap:Conclusiones sec:Sistema de detección de puntos de agarre}
El objetivo final del proyecto es la identificación de puntos de agarre de forma que las piezas puedan ser manipuladas con un robot industrial con el fin de automatizar una linea de ensamblado. La linea de ensamblaje se caracteriza por emplear una elevado diversidad de piezas que presentan diferentes formas y tamaños. Es debido a esta elevada diversidad de piezas que se ha tenido que desarrollar un sistema modular capaz de adaptarse a las diferentes tipos de piezas. El nuevo sistema consta de diferentes módulos/redes que se activan u desactivan en función del tipo de pieza que se desee manipular.

Gracias a este sistema se ha conseguido obtener el punto de agarre óptimo para las piezas grandes y se ha asumido el centro de las piezas pequeñas como un buen punto de agarre. En base a los resultados obtenidos en la fase de validación, el sistema se muestra capaz con errores pequeños y pocos \textit{outliers}. Desgraciadamente el comportamiento de las redes solo se ha podido analizar frente a un \textit{dataset} sintético pero no frente a una situación real. Para determinar con precisión el alcance y la capacidad del sistema se requiere de un \textit{dataset} real.

Pero esta falta de un \textit{dataset} real no impide que se pueda analizar la estructura del sistema y se planteen puntos de mejora. La red presenta grandes puntos fuertes como su modularidad. La cual permite la inclusión de nuevas piezas sin afectar al rendimiento de las ya implementadas y sin la necesidad de entrenar toda la red frente a un \textit{dataset} con todas las piezas. Esto permite reducir los tiempos de entrenamiento de forma sustancial durante el proceso de inclusión de nuevas piezas. Pero a su vez afecta al rendimiento global del sistema ya que implica que se debe de analizar y extraer las características de la pieza varias veces durante su análisis. Por ello se plantea la posibilidad del desarrollo de una única red capaz de aprovechar la extracción de características durante todo el proceso. Este tipo de red aumenta la complejidad del sistema pero permitirá mejorar la eficiencia. Y las últimas capas del sistema deberán de ser intercambiadas dependiendo del tipo de pieza que se vaya a identificar. De esta forma se mantiene la modularidad del sistema pero sin perder la eficiencia propia de estos sistemas.

