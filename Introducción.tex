\chapter{Introducción}
\label{chap:Introducción}
\Quote{Es siempre sabio mirar adelante, pero difícil mirar más allá de lo que puedes.}{Winston Churchill (1874--1965)} 

\Abstract{Este primer capítulo introduce a la visión artificial y demuestra el gran interés otorgado por la comunidad científica en este área. También se muestra la motivación de este proyecto, así como sus objetivos y se desarrolla la estructura del mismo.}

Los orígenes de la visión artificial surgieron en 1960 como sistemas para el reconocimiento de patrones y centrados en su posible implantación en el sector industrial debido a la gran cantidad de tareas repetitivas fácilmente automatizables \citep{50years}. A pesar de la carencia de los recursos en su momento, pero debido al gran interés del mercado estos siguieron siendo desarrollados y poco a poco implantados. Una de las primeras empresas en implantar estos sistemas fue Hitachi Labs en 1964 en Japón \cite{50years}.

Desgraciadamente estos primeros desarrollos carecían de precisión y tenían una tasa de acierto del 95\% (no se consideró suficiente para su implementación en una línea de producción). Pero esto se vería resuelto en los años venideros de forma que en la década de los 70 estos sistemas pasaron a formar parte de un gran porcentaje del sector industrial. Un claro ejemplo fue la automatización de la producción de transistores con semiconductores en 1974 por Hitachi ya que, debido a su complejidad, esta tarea no podía ser llevada a cabo por humanos de forma segura \cite{hitachi}.

En la actualidad se pueden distinguir dos tipos de visión artificial en función de sus capacidades de adaptación. En primer lugar, se encuentran los sistemas  desarrollados a partir 1960 que se caracterizan por ser programas para cumplir un único objetivo bajo unas circunstancias dadas. Estos son los menos potentes ya que no se adaptan y no saben reaccionar ante un cambio, pero son los más fáciles de desarrollar. Estos sistemas se basan en características diferenciadoras del objeto de trabajo como puede ser la forma, color, un patrón... Además, en espacios controlados pueden llegar a dar mejores resultados que sistemas más complejos y avanzados \cite{ABB}. Pero es debido a su falta de flexibilidad y a las limitaciones de características diferenciadoras de las piezas de trabajo que están siendo sustituidos.

Por otro lado, con el desarrollo de las redes neuronales y de inteligencias artificiales, se están desarrollando sistemas capaces de aprender y adaptarse a la situación a base de prueba y error. Se trata de sistemas muy modernos todavía en desarrollo que prometen traer avances como la conducción autónoma, detección de emociones en humanos, etc. Estos sistemas necesitan de grandes cantidades de bases de datos de las que aprender. Y sobre las que generar sus propias conclusiones y reglas de conducta. Su nivel de adaptabilidad y flexibilidad se ve definido por las capacidades de aprendizaje del sistema (neuronas y estructura de estas) así como de la riqueza de la base de datos. Pero gracias a los últimos avanzes, han conseguido superar la barrera de tecnología en desarrollo y se han empezado a implantar en sistemas reales.

El fin de este proyecto es la generación de un nuevo sistema de visión artificial para el reconocimiento de piezas de uso industrial. Este sistema se implantará dentro de una cadena de suministro del Grupo Antolín\textsuperscript{\textregistered} que a su vez alimentará a una linea de montaje y ensamblaje. El sistema deberá de poder identificar múltiples piezas y determinar el punto de agarre óptimo de cada pieza. Este se ve definido por sus coordenadas así como el vector normal a la superficie. De esta forma un brazo robótico con un sistema de agarre por aspiración, ventosa o \textit{soft-robotics} (varía en función de la pieza a coger) podrá recolectarlas. La multitud de herramientas de agarre así como de la necesidad de un sistema de determinación de puntos de agarre se debe a la gran variedad de piezas existentes con formas y tamaños de gran variedad (1-30 cm). El proceso completo se puede dividir en varias etapas:

\begin{enumerate}
\item Generación del pedido: en función de la demanda y de los requisitos del cliente se debe de generar una lista con todos los componentes necesarios para cada pedido.
\item Estructuración del pedido: Las piezas necesarias se encuentran distribuidas en diferentes secciones y es por ello que se debe de crear un  que determine el orden de recolección de piezas optimo que reduzca el tiempo recolección.
\item Recolección de piezas: Se trata de un proceso iterativo que se debe de realizar para cada una de las piezas que constituyen el pedido.
\begin{enumerate}[label*=\arabic*.]
\item Desplazamiento hasta la pieza. Dependiendo de la configuración del robot este sistema variará, pero el objetivo siempre será el mismo. Trasladar el robot hasta la región donde se encuentra la pieza a recolectar con el fin de poder capturarla y situarla dentro de la región de alcance.
\item Detección de la pieza: aplicando algoritmos de detección se identificará la pieza que se desea recolectar. Dentro de una misma zona se detectarán numerosas instancias de una misma pieza. Se debe de escoger la pieza/piezas mejor ubicadas y con más probabilidad de éxito.
\item Punto de agarre: tras detectar la pieza se debe de determinar como se debe de agarrar la pieza. Para ello se debe de determinar el punto de agarre óptimo y el vector normal a dicho.
\item Recolección de la pieza: se transfiere la información necesaria al robot para que este pueda recolectar la pieza a través del punto de agarre definido. El sistema de agarre a emplear dependerá del tpo de pieza.
\item Deposición y control de calidad: por último se debe de depositar la pieza dentro de la cesta que constituye el pedido. Y mediante el sistema de visión artificial se debe de comprobar que la pieza deseada ha sido correctamente depositada en la cesta.
\end{enumerate}
\item Control de calidad y trazabilidad: antes de dar por finalizado en pedido se analiza por última vez para comprobar que todas las piezas necesarias se encuentran dentro de la cesta. Se registra en el sistema el pedido y toda la información necesaria para futura trazabilidad.
\item Traslado del pedido: una vez se da por finalizado el pedido este se debe de trasladar hasta la zona de ensamblaje para comenzar el proceso de montado.
\end{enumerate}

En este proyecto nos centraremos unicamente en el desarrollo de una herramienta/sistema de visión artificial que permitirá la detección de la pieza así como la definición del punto de agarre y el vector normal a este. Este sistema se implantará para llevar a cabo la etapa 3.2 así como 3.3. El sistema también puede ser usado en las etapas 3.5 y 4 para llevar acabo las tareas de control de calidad y trazabilidad.

Una vez desarrollado el contexto de este proyecto se puede introducir la solución desarrollada. Debido a la complejidad de la salida deseada (detección más puntos de agarre) ha sido necesario desarrollar un sistema complejo compuesto por tres redes neuronales:

\begin{itemize}
\item \textit{YOLO}: se trata de la primera red neuronal empleada y desarrollada. Se encargará de detectar las piezas dentro de la zona de trabajo y se empleará como base para decidir que pieza se debe de recolectar. Es la única capa común a todas las piezas.

\item \textit{Tiny YOLO}: una red neuronal menor y menos potente pero con una mayor especialización. Esta red se encargará de determinar regiones con posibles puntos de agarre dentro de las piezas (previamente identificadas por \textit{YOLO}). Esta segunda capa es especifica para cada pieza y solo se aplicará a aquellas piezas de elevado volumen.

\item Regresor: se trata de la última capa del sistema de visión artificial. Se basa en la salida del \textit{Tiny YOLO} y determina para cada una de las posibles regiones de interés el punto de agarre óptimo y su vector normal. Esta última capa es especifica para cada pieza y solo se aplicará a aquellas piezas de elevado volumen.
\end{itemize}

Como se ha mencionado anteriormente, las capacidades de una red neuronal se ven limitadas tanto por la estructura de la red como por la riqueza de la base de datos. Y para esta aplicación concreta la obtención de dicha base de datos presenta un gran desafío debido a la gran cantidad de información que se requiere de cada imagen (identificación de las piezas, posibles puntos de agarre de cada pieza y el vector normal a dichos puntos). Es por lo que se ha optado por desarrollar a su vez una base de datos sintética que permita generar todas las imágenes e información necesarias. De esta forma se podrá automatizar el proceso de aprendizaje de la red para la introducción de nuevas piezas.